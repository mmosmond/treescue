\documentclass{beamer}

% Title page details: 
\title{Genetic signatures of evolutionary rescue}
\author{Matthew Osmond}
\institute{Department of Ecology and Evolutionary Biology\\ University of Toronto}
\date{\today}

\begin{document}

% Title page frame
\begin{frame}
    \titlepage
\end{frame}

% at least 27 yrs of theory on evol rescue: we know a lot about when and how rescue should occur
% a large and quickly growing number of experiments of increasing complexity and detail: verifying theory and discovering additional factors
% many examples of drug resistance, eg HIV, in asexual microbes: stressing an applied need to understand rescue for human health
% a big question that remains is: how common is rescue in wild sexual species? conservation angle
% evidence of the excitement around this question: a recent spike in claims for rescue in wild sexual species (amaranthus, killifish, arabidopsis, bats, etc) 
% however, still unclear how confidently we can identify rescue from (predominately) contemporary genetic data
% this is a new direction for rescue theory

% coalescent framework for detecting selection and demography
% work with Coop on signatures of rescue: difficult/impossible to differentiate sweep+bottleneck from any data, but especially summary statistics?

% what if we could see the coalecent trees themselves? does that give us more power to detect rescue? (Hejase et al 2022 say that summary statistics confound seln and demo)
% arabidopsis example (https://www.nature.com/articles/s41467-022-28800-z): used the trees, found bottleneck and sweeps but remains unclear how well their methodology can pick up rescue
% run simulations of rescue, infer trees, infer demography and selection -- can we detect rescue?

% speidel et al 2019 show that they can infer demography better than SMC in recent time, 10^2 - 10^4 generations (fig 2)
% stern et al 2019 show that they can infer allele freq and selection very well, including recent strong selection (<10^2 gens, s<0.03) from SGV or DNM (fig 7a)
% stern et al 2019 used ARGweaver but later updated the software to work with Relate, which is what Fulgione used
% hejase et al 2022: relate and SIA better at detecting seln than sum stats when selection weak and DAF low (fig 2); CLUES tends to understimate strong seln coefficients with true trees (fig 3), CLUES very much underestimates s when using inferred Relate trees (fig 4) -- note that comparing this to stern et al 2019 means that the error is coming from Relate, and that ARGweaver should do better, compare fig 4 and s5 in hejase (clues better with argweaver, but still not great and a TON of variation)

\end{document}

